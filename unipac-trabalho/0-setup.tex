% Modelo UNISINOS para teses e dissertacoes baseado no abtex2-modelo-trabalho-academico.tex, v-1.9.5 laurocesar Copyright 2012-2015 by abnTeX2 group at http://www.abntex.net.br/ 
%
% ----------------------------------------------
% abnTeX2: Modelo de Trabalho Academico (tese de doutorado, dissertacao de mestrado e trabalhos monograficos em geral) em conformidade com ABNT NBR 14724:2011: Informacao e documentacao - Trabalhos academicos - Apresentacao
% ----------------------------------------------

% ==============================================
% ||||||||||||||||||||||||||||||||||||||||||||||
% ----------------------------------------------

\documentclass[
	% -- opções da classe memoir --
	12pt,		% tamanho da fonte
	%openright,	% capítulos começam em pág ímpar (insere página vazia caso preciso)
	oneside,	% para impressão em verso e anverso. Oposto a oneside
	a4paper,	% tamanho do papel. 
	% -- opções da classe abntex2 --
	chapter=TITLE,		% títulos de capítulos convertidos em letras maiúsculas
	%section=TITLE,		% títulos de seções convertidos em letras maiúsculas
	%subsection=TITLE,	% títulos de subseções convertidos em letras maiúsculas
	%subsubsection=TITLE,% títulos de subsubseções convertidos em letras maiúsculas
	% -- opções do pacote babel --
	english,	% idioma adicional para hifenização
%	french,		% idioma adicional para hifenização - problemas com circuitikz
%	spanish,	% idioma adicional para hifenização
	brazil		% o último idioma é o principal do documento
	]{abntex2}

% ----------------------------------------------
% Pacotes de fontes... 
% ----------------------------------------------
\usepackage[utf8]{inputenc}	% Codificacao do documento (conversão automática dos acentos)
\usepackage[T1]{fontenc}	% Selecao de codigos de fonte. Afeta separação de sílabas
\usepackage{lmodern}			% Usa a fonte Latin Modern
\renewcommand{\ABNTEXchapterfont}{\fontfamily{ptm}\fontseries{sbc}\selectfont} % Familia de fontes times
%\usepackage{times}				% Usa a fonte Times
%\usepackage{palatino}			% Usa a fonte Palatino
%\usepackage{mathpazo}			% Usa a fonte Adobe Palatino
%\usepackage[scaled=.92]{helvet}	% Usa a fonte Helvetica
\usepackage{mathptmx}	% para utilização de times

% ----------------------------------------------
% Configuração das fontes
% ----------------------------------------------
% Algumas configurações de fontes para capitulos e seções tanto no texto quanto no sumário
\renewcommand{\ABNTEXchapterfont}{\bfseries}
\renewcommand{\ABNTEXchapterfontsize}{\Large}
\renewcommand{\ABNTEXpartfont}{\ABNTEXchapterfont}
\renewcommand{\ABNTEXpartfontsize}{\ABNTEXchapterfontsize}
\renewcommand{\cftpartfont}{\normalfont\bfseries}
\renewcommand{\ABNTEXsectionfont}{\bfseries}
\renewcommand{\ABNTEXsectionfontsize}{\large}
\renewcommand{\ABNTEXsubsectionfont}{\normalfont}
\renewcommand{\ABNTEXsubsectionfontsize}{\normalsize}
\renewcommand{\cftsubsectionfont}{\normalfont}
\renewcommand{\ABNTEXsubsubsectionfont}{\slshape}
\renewcommand{\cftsubsubsectionfont}{\normalfont\slshape}
\renewcommand{\ABNTEXsubsubsubsectionfont}{\bfseries}

% Para configurar mais níveis configure conforme utilizado acima e comente as duas linhas abaixo
\settocdepth{subsubsection} % configura sumário para apresentar subseções até o quarto nível
\setsecnumdepth{subsubsection} % configura para numerar subseções até o quarto nível. Subseções de quinto nível não conterão numeração.

\addto\captionsbrazil{\renewcommand{\listfigurename}{Lista de figuras}} % Altera nome da lista de ilustrações para lista de figuras

% ----------------------------------------------
% Equações com numeração sequencial
% ----------------------------------------------
\counterwithout{equation}{chapter}

% ----------------------------------------------
% Pacotes básicos 
% ----------------------------------------------
\usepackage{lastpage}		% Usado pela ficha catalografica
\usepackage{indentfirst}	% Indenta o primeiro paragrafo de cada seção.
\usepackage{color}			% Controle das cores
\usepackage{graphicx}		% Inclusão de gráficos
\usepackage{microtype} 		% para melhorias de justificação
\usepackage{array}
%\usepackage{gensymb}       % Símbolos

\usepackage{amsmath} 	%--------------------------%
\usepackage{hyperref} 	%--------------------------%
\usepackage{bibentry} 	% para inserir refs. bib. no meio do texto
		
% ----------------------------------------------
% Pacotes adicionais
% ----------------------------------------------
\usepackage{lipsum}				% para geração de dummy text
\usepackage[colorinlistoftodos, english]{todonotes}
% uso: \todo[inline, color=red!80]{texto}
\usepackage{verbatim}
\usepackage{soulutf8}
% uso: \hl{highlight} ou \st{strikeout} ou \ul{underline}
\usepackage{tabularx}
\usepackage{multirow}
\usepackage{subfig}
\usepackage{pdfpages}
\usepackage{pgfplots}
  \pgfplotsset{compat=1.12}

% Para desenho de circuitos
\usepackage{tikz}
\usepackage[american]{circuitikz}
\usepackage{siunitx}
\usepackage{colortbl}

%para usar codeblock
\usepackage{listings}
\definecolor{dkgreen}{rgb}{0,0.6,0}
\definecolor{gray}{rgb}{0.5,0.5,0.5}
\definecolor{mauve}{rgb}{0.58,0,0.82}
\lstset{frame=tb,
  language=C,
  aboveskip=3mm,
  belowskip=3mm,
  showstringspaces=false,
  columns=flexible,
  basicstyle={\small\ttfamily},
  numbers=left,
  numberstyle=\tiny\color{gray},
  keywordstyle=\color{blue},
  commentstyle=\color{dkgreen},
  stringstyle=\color{mauve},
  breaklines=true,
  breakatwhitespace=true,
  tabsize=3
} 

\usepackage{booktabs}
\usepackage{adjustbox}
\usepackage{placeins}
\usepackage{longtable}
\usepackage{caption}
\usepackage{amssymb}

% ----------------------------------------------
% Pacotes de citações
% ----------------------------------------------
\usepackage[brazilian,hyperpageref]{backref}	 % Paginas com as citações na bibl
\usepackage[alf]{abntex2cite}	% Citações padrão ABNT

% ==============================================
% CONFIGURAÇÕES DE PACOTES
% ==============================================

% ----------------------------------------------
% Configurações do pacote backref 
% usado sem a opção hyperpageref de backref
% ----------------------------------------------
\renewcommand{\backrefpagesname}{Citado na(s) página(s):~}
% Texto padrão antes do número das páginas
\renewcommand{\backref}{}
% Define os textos da citação
\renewcommand*{\backrefalt}[4]{
	\ifcase #1 %
		Nenhuma citação no texto.%
	\or
		Citado na página #2.%
	\else
		Citado #1 vezes nas páginas #2.%
	\fi}%

% ----------------------------------------------
% Espaçamentos entre linhas e parágrafos 
% ----------------------------------------------
% O tamanho do parágrafo é dado por:
\setlength{\parindent}{1.3cm}

% Controle do espaçamento entre um parágrafo e outro:
\setlength{\parskip}{0.2cm}  % tente também \onelineskip

% ----------------------------------------------
% compila o indice
% ----------------------------------------------
\makeindex
