% Modelo UNIPAC para teses e disssertações

% PARA COMEÇAR A ESCREVER NÃO SE DETENHA AOS CÓDIGOS INICIAIS DO DOCUMENTO, VOCÊ IRÁ ENTENDÊ-LOS COM O TEMPO. 
% vÁ DIRETO PARA O CAPÍTULO 1 - INTRODUÇÃO.

% Arquivo de configurações e pacotes.
\input{0-setup}

% Arquivo de dados do documento: título, autor...
% ||||||||||||||||||||||||||||||||||||||||||||||
% Informações de dados para CAPA e FOLHA DE ROSTO
% ||||||||||||||||||||||||||||||||||||||||||||||
\titulo{Título} % Não utilize o ponto final no título
\autor{Autor}
\local{Uberlândia}
\data{2018}
\orientador{Prof. Nome Sobrenome}
\coorientador{Prof. Nome Sobrenome} % comente esta linha caso nao tenha coorientador
\instituicao{%
  FACULDADE PRESIDENTE ANTÔNIO CARLOS - UNIPAC
  \par
  CURSO SUPERIOR DE TECNOLOGIA EM ANÁLISE E DESENVOLVIMENTO DE SISTEMAS}
\tipotrabalho{Tecnólogo}
% O preambulo deve conter o tipo do trabalho, o objetivo, 
% o nome da instituição e a área de concentração 

\preambulo{Qualificação apresentada como requisito parcial para obtenção do título de Tecnólogo em Análise e Desenvolvimento de Sistemas, pela Faculdade Presidente Antônio Carlos de Uberlândia - UNIPAC.}

% ----------------------------------------------
% Configurações de aparência do PDF final
% ----------------------------------------------
% alterando o aspecto da cor azul
\definecolor{blue}{RGB}{41,5,195}

% alterando o aspecto da cor cinza
\definecolor{gray}{RGB}{50,50,50}

% informações do PDF
\makeatletter
\hypersetup{
     	%pagebackref=true,
		pdftitle={\imprimirtitulo}, 
		pdfauthor={\imprimirautor},
    	pdfsubject={\imprimirpreambulo},
	    pdfcreator={LaTeX - abnTeX2 - Overleaf},
		pdfkeywords={abnt}{latex}{abntex2}{trabalho acadêmico}{unisinos}{ppgee}{mpee}{mestrado profissional},  
		colorlinks=true, % false: boxed links; true: colored links
    	linkcolor=black, % color of internal links
    	citecolor=black, % color of links to bibliography
    	filecolor=blue,  % color of file links
		urlcolor=gray,	 % color of url links
		bookmarksdepth=4
}
\makeatother

% ----------------------------------------------
% Início do documento
% ----------------------------------------------
\begin{document}

% Arquivo de elementos pré-textuais: capa, folha de rosto, ficha catalografica, errata, folha de aprovação dedicatória, agradecimentos, epígrafe, resumos, lista de abreviaturas e siglas, lista de símbolos... 
\input{2-pretextual}

% ----------------------------------------------
% ELEMENTOS TEXTUAIS
% ----------------------------------------------
\textual

% ----------------------------------------------
% Exemplo de capítulo sem numeração, mas presente no Sumário
% ----------------------------------------------
%\chapter*[Introdução]{Introdução}
%\addcontentsline{toc}{chapter}{Introdução}
% ----------------------------------------------

% ||||||||||||||||||||||||||||||||||||||||||||||
% CAP 1 - INTRODUÇÃO 
% ||||||||||||||||||||||||||||||||||||||||||||||
\chapter[Introdução]{Introdução}

\lipsum[11-12]

\chapter{Resumo}

Exemplo de código:

\begin{lstlisting}[language=Python]
import numpy as np
 
def incmatrix(genl1,genl2):
    m = len(genl1)
    n = len(genl2)
    M = None #to become the incidence matrix
    VT = np.zeros((n*m,1), int)  #dummy variable
 
    #compute the bitwise xor matrix
    M1 = bitxormatrix(genl1)
    M2 = np.triu(bitxormatrix(genl2),1) 
 
    for i in range(m-1):
        for j in range(i+1, m):
            [r,c] = np.where(M2 == M1[i,j])
            for k in range(len(r)):
                VT[(i)*n + r[k]] = 1;
                VT[(i)*n + c[k]] = 1;
                VT[(j)*n + r[k]] = 1;
                VT[(j)*n + c[k]] = 1;
 
                if M is None:
                    M = np.copy(VT)
                else:
                    M = np.concatenate((M, VT), 1)
 
                VT = np.zeros((n*m,1), int)
 
    return M
\end{lstlisting}

Exemplo de figura:

\begin{figure}[htp]
	\centering
	\caption{\label{fig:met-disc-fig01} No centro da figura é representado um logo $U_1$.} 
	\includegraphics[width = 0.8\linewidth]{images/UNIPAC.jpg}
	\legend{Fonte: Desenvolvido pelo autor.}
\end{figure}


\chapter{Conclusão}

\lipsum[11-12]


\phantompart

\bibliography{references}

% ----------------------------------------------
\end{document}
