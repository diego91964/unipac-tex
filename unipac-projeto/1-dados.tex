% ||||||||||||||||||||||||||||||||||||||||||||||
% Informações de dados para CAPA e FOLHA DE ROSTO
% ||||||||||||||||||||||||||||||||||||||||||||||
\titulo{Título} % Não utilize o ponto final no título
\autor{Autor}
\local{Uberlândia}
\data{2018}
\orientador{Prof. Nome Sobrenome}
\coorientador{Prof. Nome Sobrenome} % comente esta linha caso nao tenha coorientador
\instituicao{%
  FACULDADE PRESIDENTE ANTÔNIO CARLOS - UNIPAC
  \par
  CURSO SUPERIOR DE TECNOLOGIA EM ANÁLISE E DESENVOLVIMENTO DE SISTEMAS}
\tipotrabalho{Tecnólogo}
% O preambulo deve conter o tipo do trabalho, o objetivo, 
% o nome da instituição e a área de concentração 

\preambulo{Qualificação apresentada como requisito parcial para obtenção do título de Tecnólogo em Análise e Desenvolvimento de Sistemas, pela Faculdade Presidente Antônio Carlos de Uberlândia - UNIPAC.}

% ----------------------------------------------
% Configurações de aparência do PDF final
% ----------------------------------------------
% alterando o aspecto da cor azul
\definecolor{blue}{RGB}{41,5,195}

% alterando o aspecto da cor cinza
\definecolor{gray}{RGB}{50,50,50}

% informações do PDF
\makeatletter
\hypersetup{
     	%pagebackref=true,
		pdftitle={\imprimirtitulo}, 
		pdfauthor={\imprimirautor},
    	pdfsubject={\imprimirpreambulo},
	    pdfcreator={LaTeX - abnTeX2 - Overleaf},
		pdfkeywords={abnt}{latex}{abntex2}{trabalho acadêmico}{unisinos}{ppgee}{mpee}{mestrado profissional},  
		colorlinks=true, % false: boxed links; true: colored links
    	linkcolor=black, % color of internal links
    	citecolor=black, % color of links to bibliography
    	filecolor=blue,  % color of file links
		urlcolor=gray,	 % color of url links
		bookmarksdepth=4
}
\makeatother